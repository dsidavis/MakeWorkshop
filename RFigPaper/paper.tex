\documentclass{article}
\usepackage{graphicx}
\begin{document}

\section{Introduction}
This illustrates how Make works.
This paper is a \LaTeX{} document
with a main/master file named paper.tex.
This file displays a PNG file, fig1.png.

fig1.png is created via an R script named fig1.R.
(The names don't have to be the same, but are to help understand.)
So fig1.png depends on fig1.R.

The code in fig1.R creates the plot, but first runs (source()) the code
in setup.R.
So fig1.R depends on setup.R.

In some cases, we don't  indicate the implicit dependencies.
But in this case we do.
fig1.png depends on fig1.R and setup.R.

\begin{verbatim}
fig1.png: fig1.R setup.R

\end{verbatim}

If we wrote
\begin{verbatim}
fig1.png: fig1.R 

fig1.R: setup.R
\end{verbatim}
this would mean that fig1.R would need to be regenerated when setup.R
was changed.
But fig1.R doesn't need to be regenerated. That is something we write.
It is just that fig1.png needs to be updated when either fig1.R or setup.R change.

 

\includegraphics{fig1.png}


\section{Additional Content}

This content comes from a different file
than the main paper.tex file.
If this changes, we need to rebuild paper.pdf.

\includegraphics{fig2.png}






\end{document}

